\documentclass[12pt,largemargins]{homework}

  \newcommand{\hwname}{Phillip Janowski}
  \newcommand{\hwemail}{pajmc2@mail.umsl.edu}
  \newcommand{\hwtype}{Homework}
  \newcommand{\hwnum}{2}
  \newcommand{\hwlecture}{E}
  \newcommand{\hwsection}{01}
  \newcommand{\hwclass}{CS2700}

\usepackage{graphicx}
\usepackage{makecell}
\usepackage{enumitem}
\usepackage{multirow}
\usepackage{amsmath}
\usepackage{listings}



\date{Septemeber 6, 2018}
\begin{document}
\maketitle
\question
5.10
For the Hamming code show in Figure 5.10, show what happens when a check bit rather than a data bit is in error ?\\
\begin{center}
	\begin{tabular}{|c|c|c|c|c|c|c|c|c|c|c|c|c|}
		\hline
		Position & 12 & 11 & 10 & 9 & 8 & 7 & 6 & 5 & 4 & 3 & 2 & 1\\
		\hline 
		Bits  & D8  & D7 & D6 & D5 & C8 & D4 & D3 & D2 & C4 & D1 & C2 & C1\\
		\hline
		Block  & 0 & 0 & 1 & 1 & 1 & 1 & 0 & 0 & 1 & 1 & 1 & 1\\
		\hline
		Codes  & & & 1010 & 1001  &  & 0111 & & & & 0011 & & \\
		\hline 
	\end{tabular}
	
	\begin{tabular}{|c|c|c|}
		\hline
		Position & Code & R
		Hamming & 1111\\
		10 & 1010\\
		9 & 1001\\
		7 & 0111\\
		3 & 0011\\
		XOR'd & 1000\\
		\hline
		\end{tabesult\\
		\hlineular}\\
	It detects that there is an error in position 8.\\
\end{center}

\end{document}