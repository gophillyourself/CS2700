\documentclass[12pt,largemargins]{homework}

  \newcommand{\hwname}{Phillip Janowski}
  \newcommand{\hwemail}{pajmc2@mail.umsl.edu}
  \newcommand{\hwtype}{Homework}
  \newcommand{\hwnum}{2}
  \newcommand{\hwlecture}{E}
  \newcommand{\hwsection}{01}
  \newcommand{\hwclass}{CS2700}

\usepackage{graphicx}
\usepackage{makecell}
\usepackage{enumitem}
\usepackage{multirow}
\usepackage{amsmath}
\usepackage{listings}



\date{Septemeber 6, 2018}
\begin{document}
\maketitle
\question
5.10\\
For the Hamming code show in Figure 5.10, show what happens when a check bit rather than a data bit is in error ?\\
\begin{center}
	\begin{tabular}{|c|c|c|c|c|c|c|c|c|c|c|c|c|}
		\hline
		Position & 12 & 11 & 10 & 9 & 8 & 7 & 6 & 5 & 4 & 3 & 2 & 1\\
		\hline 
		Bits  & D8  & D7 & D6 & D5 & C8 & D4 & D3 & D2 & C4 & D1 & C2 & C1\\
		\hline
		Block  & 0 & 0 & 1 & 1 & 1 & 1 & 0 & 0 & 1 & 1 & 1 & 1\\
		\hline
		Codes  & & & 1010 & 1001  &  & 0111 & & & & 0011 & & \\
		\hline 
	\end{tabular}
	
	\begin{tabular}{|c|c|c|}
		\hline
		Position & Code\\
		Hamming & 1111\\
		10 & 1010\\
		9 & 1001\\
		7 & 0111\\
		3 & 0011\\
		XOR'd & 1000\\
		\hline
	\end{tabular}\\
	It detects that there is an error in position 8.\\
\end{center}
\question
5.11
\begin{center}
	\begin{tabular}{|c|c|c|c|c|c|c|c|c|c|c|c|c|}
		\hline
		Position & 12 & 11 & 10 & 9 & 8 & 7 & 6 & 5 & 4 & 3 & 2 & 1\\
		\hline 
		Bits  & D8  & D7 & D6 & D5 & C8 & D4 & D3 & D2 & C4 & D1 & C2 & C1\\
		\hline
		Block & 1 & 1 & 0 & 0 &   & 0 & 0 & 1 &  & 0 &  & \\
		\hline
		Code  & 1100 & 1011 &  &  &  &  & 0101 &  &  &  & \\
		\hline
		\end{tabular}
\end{center}
The check bits are in 8, 4, 2, and 1.\\
Check bit 8 calculated by values in bit numbers: 12, 11, 10 and 9 \\
$ 1 \oplus 1 \oplus 0 \oplus 0 = 0 $\\
Check bit 4 calculated by values in bit numbers: 12, 7, 6, and 5\\
$ 1 \oplus 0 \oplus 0 \oplus 1 = 0 $\\
Check bit 2 calculated by values in bit numbers: 11, 10, 7, 6 and 3\\
$ 1 \oplus 0 \oplus 0 \oplus 0 \oplus 0= 1 $\\
Check bit 1 calculated by values in bit numbers: 11, 9, 7, 5 and 3\\
$ 1 \oplus 0 \oplus 0 \oplus 1 \oplus 0 = 0 $\\
Thus, the check bits are: 0 0 1 0 \\
\question
5.12\\
Two check bits are wrong so the error is in the data\\
p2 and p4 are the two wrong bits so the corrupted bit is 6\\
00111001 should be 00011001\\
\question
5.13\\
$ 2^r \geq m + r+ 1 $\\
$ 2^r \geq 1024 + r + 1, r=10$\\
$ 2^10 \geq 1024 + 10 + 1 $\\
$ 1024 \geq 1024 +10 +1 $ false\\
$ 2^11 \geq 1024 +11 +1 $\\
$ 2048 \geq 1036 $ true \\
A minimum of 11 check bits needed for 1024-bit Hamming code error correction\\

\question
6.5\\
\begin{alphaparts}
	\item 
	Average Seek Time\\
	Becuase we start at track 0 of 30,000, the most we would have to traverse is 29,999 tracks\\
	Average track number is $ \frac{29,999}{2} = 14,999.5$\\
	Average track seek time with a $ \frac{100 tracks}{1 ms} $\\
	$ 14,999.5/100 = $ 149.995 ms\\
	\item 
	Average Rotational Latency\\
	$ \frac{7200 rev}{minute} * \frac{Minute}{60 sec} * \frac{Second}{1000ms} = 8.33ms$\\
	Because the average track is in the middle of any two given tracks we divide by 2\\
	$ \frac{8.333}{2} = 4.16\overline{6} ms$\\
	\item 
	Transfer time per sector\\
	$ \frac{600 sectors}{track} $ or $ \frac{600 sectors}{revolution} \rightarrow \frac{8.333ms}{revolution} * \frac{revolution}{600 track} = 0.01389 ms$\\
	\item
	Average Total Satisfy Time\\
	$ 149.995 + 4.166 0.01389 = 154 ms$\\
	
\end{alphaparts}

\question
6.7\\
\begin{alphaparts}
	\item 
	Transfer time sector 1 on track 8 to sector 1 on track 9\\
	$ \frac{60,000}{360} \rightarrow 16.67ms $\\
	$ \frac{16.67}{32 tracks} = .52 ms $\\
	$ 16.7 * \frac{31}{32} = 16.2 ms $ rotational delay\\
	$ .52 + 16.2 + .52 = 17.24 ms $\\
	\item 
	Transfer all sectors from track 8 to track 9\\
	The write will start on the 5th sector of track 9\\
	$ .52 ms * 4 = 2.08 ms $\\
	$ 16.7 * 2 + 2.08 = 35.48 ms $\\
\end{alphaparts}
\end{document}