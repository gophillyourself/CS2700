\documentclass[12pt,largemargins]{homework}

  \newcommand{\hwname}{Phillip Janowski}
  \newcommand{\hwemail}{pajmc2@mail.umsl.edu}
  \newcommand{\hwtype}{Homework}
  \newcommand{\hwnum}{1}
  \newcommand{\hwlecture}{E}
  \newcommand{\hwsection}{01}
  \newcommand{\hwclass}{CS2700}

\usepackage{multirow}
\usepackage{amsmath}
\usepackage{listings}

\date{Septemeber 6, 2018}
\begin{document}
\maketitle
\normalsize
	\question
	2.2 Consider two different machines, with two different instruction sets, both of which
have a clock rate of 200 MHz. The following measurements are recorded on the two
machines running a given set of benchmark programs: \\
	\begin{tabular}{|l|c|c|}
		\hline
		Instruction Type & Instruction Count (millions) & Cycles Per Instruction \\
		\hline 
		Machine A & & \\
	     Arithmetic and logic & 8 & 1 \\
	     Load and store & 4 & 3 \\
	     Branch & 2 & 4 \\
	     Others & 4 & 3\\
	     \hline
	     Machine B & & \\
	     Arithmetic and logic & 10 & 1 \\
	     Load and store & 8 & 2  \\
		 Branch & 2 & 4 \\
	     Others & 4 & 3\\
	     \hline
	     
	\end{tabular}
		\begin{alphaparts}

	\item 
	Determine the effective CPI, MIPS rate, and execution time for each machine. \\		     		Solution: \\
	$CPU_A =\dfrac{\sum CPI_i * I_i}{I_c}$
	$= \dfrac{(8*1 +4 *3 + 2*4 + 4*3)*10^6}{(8+4+2+4)*10^6} = 2.\overline{22}$\\
	$MIPS_A=\dfrac{f}{CPI_A*10^6} = \dfrac{200*10^6}{2.22*10^6} = 90$ \\
	$CPU_A=\dfrac{I_c*CPI_A}{f}=\dfrac{18*10^6*2.2}{200*10^6} =.2$ seconds\\
	$CPI_B=\dfrac{\sum CPI_i * I_i}{I_c} 
	= \dfrac{(10*1+8*2+2*4+4*3)*10^6}{(10+8+2+4)*10^6} = 1.91\overline{6}$ \\
	$MIPS_B = \dfrac{f}{CPI_B*10^6} =\dfrac{200*10^6}{1.92*10^6} = 104$ \\
	$CPU_B=\dfrac{I_c*CPI_B}{f}=\dfrac{24*10^6*1.92}{200*10^6} = .23$ seconds
	\item 
	Comment on the results.\\
	Solution: \\
	Machine B takes more CPU time to finish its benchmark even though it has a higher MIPS that Machine A

	\end{alphaparts}
\question
2.5 The following table, based on data reported in the literature [HEAT84], shows
the execution times, in seconds, for five different benchmark programs on three
machines.\\
\begin{center}

\begin{tabular}{|c|c|c|c|}
	\hline
	\multirow{2}{*}{Benchmark} & \multicolumn{3}{|c|}{Processor} \\
	\cline{2-3}
	& R & M & Z \\
	\hline
	E & 417 & 244 & 134 \\
	\hline
	F & 83 & 70 & 70 \\
	\hline
	H & 66 & 153 & 135\\
	\hline 
	I & 39449 & 35527 & 66000\\
	\hline
	K & 771 & 368 & 369 \\
	\hline
\end{tabular}
\end{center}
\begin{alphaparts}
\item
Compute the speed metric for each processor for each benchmark, normalized to
machine R. That is, the ratio values for R are all 1.0. Other ratios are calculated
using Equation (2.5) with R treated as the reference system. Then compute the
arithmetic mean value for each system using Equation (2.3). This is the approach
taken in [HEAT84].\\
Solution: \\
\begin{center}
Arithmetic Mean $= \dfrac{1}{n} \sum_{i=1}^n x_i$ \\
Normal to R\\
\begin{tabular}{|c|c|c|c|}
	\hline
	& R & M & Z\\
	\hline
	E & $417/417 = 1$ & $417/244 = 1.71$ & $417/134 = 3.11$\\
	\hline
	F & $83/83 = 1$ & $83/70 = 1.186$ & $80/70 = 1.186$\\
	\hline
	H & $66/66 = 1$ & $66/153 = .431$ & $ 66/135 = .4\overline{88}$\\
	\hline
	I & $ 39449/39449 = 1$ & $ 39449/35527 = 1.11$ & $39449/66000 = .598$ \\
	\hline
	K & $772/772 = 1$ & $772/368 = 2.098$ & $772/369=2.092$\\
	\hline 
	AM & 
\end{tabular}
\end{center}
\item
Repeat part (a) using M as the reference machine. This calculation was not tried in
[HEAT84].\\
\item
Which machine is the slowest based on each of the preceding two calculations?\\
\item 
Repeat the calculations of parts (a) and (b) using the geometric mean, defined in
Equation (2.6). Which machine is the slowest based on the two calculations?\\
\end{alphaparts}
\end{document}